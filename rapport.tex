\documentclass[a4paper]{llncs}

\usepackage[utf8]{inputenc}
\usepackage{tabto}

\usepackage{natbib}
\bibliographystyle{apalike-fr}

\usepackage{amssymb}
\setcounter{tocdepth}{3}
\usepackage{graphicx}

\usepackage[french]{babel} % Pour adopter les règles de typographie française
\usepackage[T1]{fontenc} % Pour que les lettres accentuées soient reconnues

\newcommand{\keywords}[1]{\par\addvspace\baselineskip
\noindent\keywordname\enspace\ignorespaces#1}

\begin{document}


\mainmatter 

\title{Résolution du SSCFLP}

\author{Dumez Dorian}

\institute{Universite de Nantes}

\tocauthor{{}}

\maketitle

\medskip

\begingroup
\let\clearpage\relax
\tableofcontents
\addcontentsline{toc}{section}{Introduction}
\endgroup

\medskip
\medskip

\section*{Introduction}

\section{Problèmes étudiés}

La première, et plus simple, version de notre problème est le UFLP. C'est à dire la localisation de service sans capacité. On se permet alors de supposer qu'un dépos peux alimenter un nombre infini de clients, on ne quantifie donc pas la demande des clients.

Le modèle de programmation lineaire est : \\
$min z = \sum \limits_{j \in J} f_j x_j + \sum \limits_{i \in I} \sum \limits_{j \in J} c_{ij} y_{ij}$\\
s.c.
\begin{equation}
 \forall i \in I \forall  \in J : y_{ij} - x_j \leqslant 0
\end{equation}
\begin{equation}
 \forall i \in I : \sum \limits_{j \in J} y_{ij} \geqslant 1
\end{equation}
La contrainte 1 exprime le fait que le depos j peux alimenter le clien i seulement si il est ouvert. La deuxieme exprime le fait que tout les clients doivent etre couvert.

\section{Instances choisies}

Toutes les instances viennent du site http://www-eio.upc.es/~elena/sscplp/index.html.\\

La première instance, p1, est une petite instance normale, elle ne comprend que 20 clients et 10 possibilite de facilités. Les coûts d'ouverture sont relativement grand par rapport aux coûts de connexions et les capacités sont aussi grandes par rapport aux demandes.\\

L'instance p2, petite, possède des coût de connexions elevé par rapport aux coûts d'ouvertures.\\

L'instance p3 est aussi de petite taille mais presente des facilités avec des coûts d'ouvertures très elevés.\\

L'instance p6, toujours petite, mais nous propose des facilités peu chere mais avec de petites capacités par rapport aux demandes.\\

Suivant le même shémat on effectue une monté en charge avec les instances p7(normale), p9(petite et pas chere), p10(facilité chères) et p13(coûts de connexions elevé). En effet celles ci on 30 clients pour 15 possibilité de facilités.\\

Toujours selon le même principe on passe à des instances à 50 clients et 20 sites possibles. On utilise les instances p26(normale), p33(coûts d'ouvertures elevés), p31(facilités peux chères) et p30 (coûts de connexions elevés).\\

\section*{Bibliographie}
Sources des instances : http://www-eio.upc.es/~elena/sscplp/index.html.\\

\end{document}
